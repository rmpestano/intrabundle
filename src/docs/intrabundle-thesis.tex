%
% exemplo genérico de uso da classe iiufrgs.cls
% $Id: iiufrgs.tex,v 1.1.1.1 2005/01/18 23:54:42 avila Exp $
%
% This is an example file and is hereby explicitly put in the
% public domain.
%
\documentclass[cic,tc,english]{iiufrgs} % pode-se usar 'dipl' em vez de 'tc'
% um tipo específico de monografia pode ser informado como parâmetro opcional:
%\documentclass[tese]{iiufrgs}
% monografias em inglês devem receber o parâmetro `english':
%\documentclass[diss,english]{iiufrgs}
% a opção `openright' pode ser usada para forçar inícios de capítulos
% em páginas ímpares
% \documentclass[openright]{iiufrgs}
% para gerar uma versão somente-frente, basta utilizar a opção `oneside':
% \documentclass[oneside]{iiufrgs}
\usepackage[T1]{fontenc}        % pacote para conj. de caracteres correto
\usepackage[utf8]{inputenc}   % pacote para acentuação
\usepackage{graphicx}           % pacote para importar figuras
%\usepackage[outdir=./]{epstopdf}
%\usepackage{epstopdf} 
\usepackage{times}              % pacote para usar fonte Adobe Times
%\usepackage{mathptmx}          % p/ usar fonte Adobe Times nas fórmulas

%
% Informações gerais
%
\title{Um Exemplo de Monografia do Instituto de Informática da UFRGS}

\author{Flaumann}{Fritz Gutenberg}
% alguns documentos podem ter varios autores:
%\author{Flaumann}{Frida Gutenberg}
%\author{Flaumann}{Klaus Gutenberg}

% orientador e co-orientador são opcionais (não diga isso pra eles :))
\advisor[Prof.~Dr.]{Lamport}{Leslie}
%\coadvisor[Prof.~Dr.]{Knuth}{Donald Ervin}

% a data deve ser a da defesa; se nao especificada, são gerados
% mes e ano correntes
%\date{maio}{2001}

% o nome do curso pode ser redefinido (ex. para TCs)
%\course{Curso de Especialização em Cachaça}

% o local de realização do trabalho pode ser especificado (ex. para TCs)
% com o comando \location:
%\location{Itaquaquecetuba}{SP}

% itens individuais da nominata podem ser redefinidos com os comandos
% abaixo:
% \renewcommand{\nominataReit}{Prof\textsuperscript{a}.~Wrana Maria Panizzi}
% \renewcommand{\nominataReitname}{Reitora}
% \renewcommand{\nominataPRE}{Prof.~Jos{\'e} Carlos Ferraz Hennemann}
% \renewcommand{\nominataPREname}{Pr{\'o}-Reitor de Ensino}
% \renewcommand{\nominataPRAPG}{Prof\textsuperscript{a}.~Joc{\'e}lia Grazia}
% \renewcommand{\nominataPRAPGname}{Pr{\'o}-Reitora Adjunta de P{\'o}s-Gradua{\c{c}}{\~a}o}
% \renewcommand{\nominataDir}{Prof.~Philippe Olivier Alexandre Navaux}
% \renewcommand{\nominataDirname}{Diretor do Instituto de Inform{\'a}tica}
% \renewcommand{\nominataCoord}{Prof.~Carlos Alberto Heuser}
% \renewcommand{\nominataCoordname}{Coordenador do PPGC}
% \renewcommand{\nominataBibchefe}{Beatriz Regina Bastos Haro}
% \renewcommand{\nominataBibchefename}{Bibliotec{\'a}ria-chefe do Instituto de Inform{\'a}tica}
% \renewcommand{\nominataChefeINA}{Prof.~Jos{\'e} Valdeni de Lima}
% \renewcommand{\nominataChefeINAname}{Chefe do \deptINA}
% \renewcommand{\nominataChefeINT}{Prof.~Leila Ribeiro}
% \renewcommand{\nominataChefeINTname}{Chefe do \deptINT}

% A seguir são apresentados comandos específicos para alguns
% tipos de documentos.

% Relatório de Pesquisa [rp]:
% \rp{123}             % numero do rp
% \financ{CNPq, CAPES} % orgaos financiadores

% Trabalho Individual [ti]:
% \ti{123}     % numero do TI
% \ti[II]{456} % no caso de ser o segundo TI

% Trabalho de Conclusão [tc]:
% além de definir explicitamente o nome do curso (\course) e o local
% de realização (\location), é necessário redefinir a nominata,
% pois as informações necessárias dependem do curso. Ex.:
%\renewcommand{\nominata}{
%        UNIVERSIDADE FEDERAL DO RIO GRANDE DO SUL\\
%        Reitora: Prof\textsuperscript{a}.~Wrana Maria Panizzi\\
%        Pró-Reitor de Ensino: Prof.~José Carlos Ferraz Hennemann\\
%        Diretor do Instituto de Informática: Prof.~Philippe Olivier Alexandre Navaux\\
%        Coordenador do curso: Prof.~Seu Creysson\\
%        Bibliotecária-chefe do Instituto de Informática: Beatriz Regina Bastos Haro
%}

% Monografias de Especialização [espec]:
% \espec{Redes e Sistemas Distribuídos}      % nome do curso
% \coord[Profa.~Dra.]{Weber}{Taisy da Silva} % coordenador do curso
% \dept{INA}                                 % departamento relacionado

%
% palavras-chave
% iniciar todas com letras minúsculas, exceto no caso de abreviaturas
%
\keyword{formatação eletrônica de documentos}
\keyword{\LaTeX}
\keyword{ABNT}
\keyword{UFRGS}

%
% inicio do documento
%
\begin{document}

% folha de rosto
% às vezes é necessário redefinir algum comando logo antes de produzir
% a folha de rosto:
% \renewcommand{\coordname}{Coordenadora do Curso}
\maketitle

% dedicatoria
\clearpage
\begin{flushright}
\mbox{}\vfill
{\sffamily\itshape
``If I have seen farther than others,\\
it is because I stood on the shoulders of giants.''\\}
--- \textsc{Sir~Isaac Newton}
\end{flushright}


% agradecimentos
\chapter*{Acknowledgments}
I would like to thank my family for all the support and love. I would like to thank my wife and daughter, they gave me all the strength needed in the hard times. I would like to say thanks to my brother Marcos Mauricio Pestano who always inspired me. I would like to thank my mother for all the support and love. I would like to say thanks to a hero, who unfortunately left us during this work, which is my Father Celso Morales Pestano, a great example of person and source of inspiration.

I would like to thank the Federal University of Rio Grande do Sul (UFRGS) and Informatics Institute for providing an education of excellence. I'd like to say special thanks to Prof. Dr. Cláudio Fernando Resin Geyer and João Claudio Américo for all patience and support on this work. Also would like to say thanks to Prof. Dr. Didier Donzes for all ideas and all kind of information he provided during this work.

Finally I would like to say thanks to everyone who helped me to reach here.
   


\begin{englishabstract}{}
As aplicações de software hoje em dia estão cada vez mais complexas, maiores, dinâmicas e mais difíceis de manter. Uma maneira de superar as complexidades dos sistemas modernos é através de aplicações modulares as quais são dividas em partes manores que colaboram entre si para resolver problemas maiores, o famoso \emph{dividir para conquistar}. Outro aspecto importante na industria de software que ajuda à construir aplicações grandes é o conceito de qualidade de software já que é sabido que quanto maior a qualidade do software mais facil de mante-lo e evolui-lo a logo prazo será.

The Open Services Gateway Initiative(OSGi) é o \emph{padrão de fato} para se criar aplicações modulares em java porém não existe forma automatizada de se medir a qualidade de sistemas OSGi. No ambito de aplicações java existem diversas metricas de qualidade e ferramentas para medir a qualidade de softwar mas quando entramos no contexto de aplicações modulares, onde as métricas conhecidas não se encaixam ou não existem, ficamos sem opções. 

Neste trabalho será apresentada uma ferramenta chamada \emph{Intrabundle} que analisa projetos OSGi a mede sua qualidade. Ainda seram propostas métricas de qualidade baseadas em boas práticas conhecidas do mundo OSGi que serão aplicadas em 10 projetos reais que variam em tamanho, equipes e domínio.
\end{englishabstract}


\begin{abstract}{}
 Todays software applications are becoming more complex, bigger, dynamic and harder to maintain. One way to overcome nowadays complexities is to build modular applications so we can divide the problems into small blocks which collaborate to solve bigger problems, the so called \emph{divide to conquer}. Another important aspect in the software industry that helps building large applications is the notion of software quality because its well known that higher quality softwares are easier to maintain and evolve at long term.

The Open Services Gateway Initiative(OSGi) is the \emph{de facto} standard for building Java modular applications but there is no automated way to measure the quality of an OSGi system. In the context of Java applications there are many well known quality metrics to measure application's quality but when we move to Java modular applications where standard quality metrics does not fit or even exist we run out of options. 

In this work we will present a tool called \emph{intrabundle} which analyses OSGi applications and measure their quality. We also propose 6 metrics based on good practices inside OSGi world and apply them on top of 10 real OSGi projects which vary in size, teams and domain of solution.
\end{abstract}



% lista de figuras
\listoffigures

% lista de tabelas
%\listoftables

% lista de abreviaturas e siglas
% o parametro deve ser a abreviatura mais longa
\begin{listofabbrv}{SPMD}
        \item[SMP] Symmetric Multi-Processor
        \item[NUMA] Non-Uniform Memory Access
        \item[SIMD] Single Instruction Multiple Data
        \item[SPMD] Single Program Multiple Data
        \item[ABNT] Associação Brasileira de Normas Técnicas
\end{listofabbrv}

% sumario
\tableofcontents

% idem para a lista de símbolos
%\begin{listofsymbols}{$\alpha\beta\pi\omega$}
%       \item[$\sum{\frac{a}{b}}$] Somatório do produtório
%       \item[$\alpha\beta\pi\omega$] Fator de inconstância do resultado
%\end{listofsymbols}

% aqui comeca o texto propriamente dito

%introdução
% introducao
\chapter{Introduction}
 This chapter will drive the reader through the context and motivation of this work followed by the objectives and later the organization of this text is presented.  
 

\section{Context}

% stating how important quality is % 
One of the pillars of sustainable software development is its quality which can basically be defined as functional or non-functional where the first focuses on how the software meets its specification and how it works accordingly and the second is aimed on how well the software is structured, we can generalize the first as being \emph{external quality} and second as \emph{internal quality}. To measure external quality there is the need to execute the software, also known as \emph{dynamic analysis}, either by an end user accessing the system or an automated process like for example functional testing. There is no known way to assure functional quality without executing the software. 

Internal quality however can be verified by either statical analysis that is mainly the inspection of the source code itself or by executing the software like for example automated whitebox testing also known as structural testing(Beizer, 1995).   

% OSGi and its importance %
A well known and successful and way to structure software architecture is to modularize its components. In the Java ecosystem although there is a moving to modularize the JDK with the project Jigsaw(TODO reference) for now the only practical working solution for modular Java applications is OSGi, a component-based and service-oriented framework for building Java modular applications which is the \emph{de facto} standard solution for this kind of software since early 2000's. 

In the context of Java modular applications and OSGi there is no way to measure software internal quality which is the main objective of this work.          


\section{Objectives}

This work is focused on internal OSGi projects quality mainly because of the following facts:
\begin{enumerate}
  \item there is no known standard way to measure OSGi internal quality.
  \item We already have tools and approaches to measure non OSGi projects internal and external quality.
  \item For OSGi applications measuring external quality the classical approaches like automated testing are sufficient.
\end{enumerate}
For measuring OSGi qualities we first will create the metrics based on good practices in the development of OSGi systems so in a second moment we can apply those metrics to real OSGi projects using a tool called \emph{intrabundle} which was created during this work and also will be presented here. In the end we will analyze the resulting output of intrabundle and analyzed projects qualities and conclude if the metrics we created have value for measuring Java modular applications or not.  

 
\section{Organization}

This text is organized in the following way. First chapter defines the context, motivation and objectives of this work. The second chapter will introduce the main concepts in the area of software quality like quality measurement, quality metrics, program analysis and quality analysis tools. The third chapter will present Java and OSGi, how standard Java and OSGi are different in respect to quality metrics and why we need different metrics for OSGi(TODO - depending it will be merged into chapter two). The fourth chapter presents Intrabundle, a OSGi code introspection tool to measure internal quality, we will see how Intrabundle works, what kind of information it extracts and what metrics it is applying. The fifth chapter will analyses the results intrabundle produces and validate them to decide if this work has a valid contribution or not. The last chapter will present the conclusions and future work on this subject. 
   
 
% e aqui vai a parte principal
 
% \chapter{Estado da arte}
\chapter{State of Art}
This chapter presents an overview of the concepts and technologies that were studied and used on the development of this work. 
In section \textit{2.1 - Software Quality}, will be presented general aspects of software quality such as \textit{Quality Measurement},  \textit{software metrics}, \textit{Program Analysis} and some tools that are used in this area.  

Section \textit{2.2 - Java and OSGi} will introduce OSGi a framework for build service oriented Java modular applications as well the motivation 
behind this solution and why standard quality metrics aren't sufficient for this king of application. 


\section{Software Quality}



There are two main motivations to perform continuous software quality analysis that are \textbf{Risk management} and \textbf{Cost management}  


- functional quality(performed via automated testing)
- structural quality(\textbf{this is where our work shines})

\subsection{Quality Measurement}

\subsubsection{Code Based Analysis}
\subsubsection{Efficiency}
\subsubsection{Maintainability}
\subsubsection{Other kinds of software Quality Measurement}

\subsection{Software Metric}
\subsubsection{Common Software Measurements}

\subsection{Program Analysis}
Program analysis is the process of automatically analyzing the behavior of computer programs. Two main approaches in program analysis are \textbf{static program analysis} and \textbf{dynamic program analysis}. Main applications of program analysis are program correctness and program optimization.
\subsubsection{Dynamic Program Analysis}
\subsubsection{Static Program Analysis}

\subsection{Quality Analysis Tools}
This section will list most used code quality analysis tools.

\section{Java and OSGi}
I the context of Java modular applications...

% \chapter{Mais estado da arte}
\chapter{More state of art}
more state of art

% \chapter{A minha contribuição}
\chapter{Intrabundle - An OSGi Bundle Introspection Tool}

\section{Introduction}

\section{Design Decisions}
To analyze large code bases of OSGi projects which can vary from KLOCs to thousands of KLOCs we needed a lightweight approach with the following functional requirements:

- 

The following alternatives were evaluated:

-

\section{JBoss Forge}

\section{Implementation Overview}

\section{Collecting Bundle Data}

\section{Metrics Calculation}

\section{Intrabundle Quality}
In this section we will see how intrabundle quality is managed and how some concepts of \textit{chapter 2 - State of art} were applied to the project.
\subsection{Internal quality}
Intrabundle internal is managed by PMD and JaCoCo. PMD is an static analysis tool and JaCoCo a dynamic analysis one. Both were presented at Chapter two in section \textit{Quality Analysis Tools} with the objective to guarantee non functional requirements.

\subsubsection{Example}
 A PMD was already illustrated at Chapter 2 as an example of static analysis tool. JaCoCo is used to calculate code coverage to track files and methods that automated tests are covering. Figure 3.1 shows JaCoCo code coverage report for Intrabundle:

\begin{figure}[h]
\caption{Intrabundle code coverage}
\includegraphics[scale=0.5]{intrabundle-code-coverage}
\end{figure}

\FloatBarrier

\subsection{External quality}
Intrabunde external quality is assured by automated whitebox tests so we can verify if Intrabundle is working as expected, if it meets its functional requirements.

\subsubsection{Example}
Intrabundle performs 62 \textbf{integration tests}, which can be defined as automated tests aimed to detect any inconsistencies between the software units that are integrated together, to guarantee its external quality. In this kind of automated tests the system must be running and in case of Intrabundle we also need the Forge runtime up and running during tests and that is done by Arquillian \citep{dan 2011}, an integration test platform. Figure 3.2 shows the result of integration tests execution:

\begin{figure}[h]
\caption{Intrabundle external tests}
\includegraphics[scale=0.5]{intrabundle-external-quality}
\end{figure}

\FloatBarrier
% \chapter{Prova de que a minha contribuição é válida}
\chapter{Bundle Introspection Results}

This chapter will make a deep analysis if results and prove that my contribution is valid(or not)
% \chapter{Conclusão}
\chapter{Conclusão}
Capítulo para conclusão






% referencias
% aqui será usado o environment padrao `thebibliography'; porém, sugere-se
% seriamente o uso de BibTeX e do estilo abnt.bst (veja na página do
% UTUG)
%
% observe também o estilo meio estranho de alguns labels; isso é
% devido ao uso do pacote `natbib', que permite fazer citações de
% autores, ano, e diversas combinações desses
\begin{thebibliography}{este-parametro-nao-eh-usado-pelo-estilo-ABNT}

\bibitem[ANDREWS, 1991]{Andrews:CP-91} ANDREWS,
  G.~R\@. \textbf{Concurrent programming}: principles and
  practice. Redwood~City, USA: Benjamin/Cummings, 1991. 637p.
  
\bibitem[ASSENMACHER et~al.(1993)ASSENMACHER; BREITBACH; BUHLER;
  H{\"U}BSCH; SCHWARZ]{Assenmacher:Panda-ECOOP93} ASSENMACHER, H.;
  BREITBACH, T.; BUHLER, P.; H{\"U}BSCH, V.; SCHWARZ, R\@.
  Panda---supporting distributed programming in {C}++. In: EUROPEAN
  CONFERENCE ON OBJECT-ORIENTED PROGRAMMING, 7., 1993, Kaiserslautern,
  Germany. \textbf{Proceedings{\ldots}} Berlin: Springer-Verlag, 1993.
  p.361--383. (Lecture Notes in Computer Science, v.707).

\bibitem[BAKER; SMITH, 1996]{Baker:PP-96} BAKER, L.; SMITH,
  B.~J\@. \textbf{Parallel programming}. New~York: McGraw-Hill,
  1996. 381p.

\bibitem[CAROMEL; KLAUSER; VAYSSIERE, 1998]{Caromel:TSC-CPE-10-11-98}
  CAROMEL, D.; KLAUSER, W.; VAYSSIERE, J\@. Towards seamless computing
  and metacomputing in {J}ava.  \textbf{Concurrency: Practice and
  Experience}, West~Sussex, v.10, n.11--13, p.1043--1061,
  Sept./Nov.~1998.

\bibitem[FURMENTO; ROUDIER; SIEGEL, 1995]{Furmento:PDC-95} FURMENTO,
  N.; ROUDIER, Y.; SIEGEL, G\@. \textbf{Parall{\'e}lisme et
  distribution en {C}++}: une revue des langages existants. Valbonne,
  FR: I3S, Universit\'{e} de Nice Sophia-Antipolis, 1995. (RR~95-02).

\bibitem[INSTITUTE OF ELECTRICAL AND ELECTRONIC ENGINEERS,
  1995]{IEEE:Pthreads-95} INSTITUTE OF ELECTRICAL AND ELECTRONIC
  ENGINEERS\@. \textbf{Information Technology---Portable Operating
  System Interface (POSIX), Threads Extension [C Language]},
  \mbox{IEEE}~1003.1c-1995.  New~York, 1995.

\bibitem[SILBERSCHATZ; PETERSON; GALVIN, 1991]{Silberschatz:OSC-3-91}
  SILBERSCHATZ, A.; PETERSON, J.~L.; GALVIN, P.~B\@. \textbf{Operating
  system concepts}. 3.ed.  Reading, USA: Addison-Wesley, 1991. 696p.

\bibitem[UTUG(2001)UTUG]{UTUG:Homepage-01} UTUG\@. \textbf{Página do grupo
  de usuários {\TeX} da {UFRGS}}. Disponível em:
  $<$http://www.inf.ufrgs.br/utug$>$. Acesso em: maio 2001.

\bibitem[WILSON, 2001]{Wilson:MME-01} WILSON, P.~C\@. \textbf{Um
  método ótimo para o preparo de café em laboratório baseado na
  reciclagem de filtros}. 2001. 123p.  Disserta{\c{c}}{\~a}o (Mestrado
  em Ci{\^e}ncia da Computa{\c{c}}{\~a}o) --- Instituto de
  Inform{\'a}tica, Universidade Federal do Rio Grande do Sul,
  Porto~Alegre.

\end{thebibliography}

\end{document}
