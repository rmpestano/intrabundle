\begin{englishabstract}{}{OSGi. java. quality. metrics. modularity. intrabundle}
As aplicações de software hoje em dia estão cada vez mais complexas, maiores, dinâmicas e mais difíceis de manter. Uma maneira de superar as complexidades dos sistemas modernos é através de aplicações modulares as quais são divididas em partes menores que colaboram entre si para resolver problemas maiores, o famoso \emph{dividir para conquistar}. Outro aspecto importante na industria de software que ajuda a construir aplicações grandes é o conceito de qualidade de software já que é sabido que, quanto maior a qualidade do software, mais facil de mante-lo e evolui-lo a longo prazo será.

The Open Services Gateway Initiative(OSGi) é uma solução bastante popular para se criar aplicações modulares em Java porém é muito dificil medir a qualidade interna de sistemas OSGi devido a suas caracteristicas particulares como arquitetura orientada a serviços e componentes assim como modularidade intrínseca. 

Neste trabalho será apresentada uma ferramenta chamada \emph{Intrabundle} que analisa projetos OSGi e mede sua qualidade interna. A ferramenta extrai informações úteis que são específicas desse tipo de projeto e organiza os dados extraídos em relatórios em diversos formatos.

Ainda foram propostas métricas de qualidade baseadas em boas práticas conhecidas do mundo OSGi que serão aplicadas em 10 projetos reais que variam em tamanho, equipes e domínio.
\end{englishabstract}
