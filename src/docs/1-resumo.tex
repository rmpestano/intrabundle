\begin{englishabstract}{}{OSGi. java. quality. metrics. modularity. intrabundle}
As aplicações de software hoje em dia estão cada vez mais complexas, maiores, dinâmicas e mais difíceis de manter. Uma maneira de superar as complexidades dos sistemas modernos é através de aplicações modulares as quais são dividas em partes manores que colaboram entre si para resolver problemas maiores, o famoso \emph{dividir para conquistar}. Outro aspecto importante na industria de software que ajuda à construir aplicações grandes é o conceito de qualidade de software já que é sabido que quanto maior a qualidade do software mais facil de mante-lo e evolui-lo a logo prazo será.

The Open Services Gateway Initiative(OSGi) é o \emph{padrão de fato} para se criar aplicações modulares em java porém não existe forma automatizada de se medir a qualidade de sistemas OSGi. No ambito de aplicações java existem diversas metricas de qualidade e ferramentas para medir a qualidade de software mas quando entramos no contexto de aplicações modulares, onde as métricas conhecidas não se encaixam ou não existem, por exemplo dependência entre módulos, ficamos sem opções. 

Neste trabalho será apresentada uma ferramenta chamada \emph{Intrabundle} que analisa projetos OSGi a mede sua qualidade. Ainda seram propostas métricas de qualidade baseadas em boas práticas conhecidas do mundo OSGi que serão aplicadas em 10 projetos reais que variam em tamanho, equipes e domínio.
\end{englishabstract}
