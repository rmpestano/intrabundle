\chapter{State of Art}
This chapter presents an overview of the concepts and technologies that were studied and used on the development of this work. 
In section \textit{2.1 - Software Quality}, will be presented general aspects of software quality such as \textit{Quality Measurement},  \textit{software metrics}, \textit{Program Analysis} and some tools that are used in this area.  

Section \textit{2.2 - Java and OSGi} will introduce OSGi a framework for build service oriented Java modular applications as well the motivation 
behind this solution and why standard quality metrics aren't sufficient for this king of application. 


\section{Software Quality}



There are two main motivations to perform continuous software quality analysis that are \textbf{Risk management} and \textbf{Cost management}  


- functional quality(performed via automated testing)
- structural quality(\textbf{this is where our work shines})

\subsection{Quality Measurement}

\subsubsection{Code Based Analysis}
\subsubsection{Efficiency}
\subsubsection{Maintainability}
\subsubsection{Other kinds of software Quality Measurement}

\subsection{Software Metric}
\subsubsection{Common Software Measurements}

\subsection{Program Analysis}
Program analysis is the process of automatically analyzing the behavior of computer programs. Two main approaches in program analysis are \textbf{static program analysis} and \textbf{dynamic program analysis}. Main applications of program analysis are program correctness and program optimization.
\subsubsection{Dynamic Program Analysis}
\subsubsection{Static Program Analysis}

\subsection{Quality Analysis Tools}
This section will list most used code quality analysis tools.

\section{Java and OSGi}
I the context of Java modular applications...
