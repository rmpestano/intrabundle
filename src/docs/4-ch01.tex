\chapter{State of Art}
This chapter presents an overview of the concepts and technologies that were studied and used on the development of this work. Section 2.1(TODO reference subsection) introduces general \textit{Software Quality}, 2.2 present the concepts of code \textit{Quality Analysis}, 2.3 introduces the concept of \textit{Software Metric}, 2.4 shows the concept of \textit{Program Analysis} and section 2.5 lists well known \textit{Code Quality Analysis Tools}.  


\section{Introduction}
This section will talk about general quality analysis 



\section{Software Quality}
This section will talk about software quality
- functional quality(performed via automated testing)
- structural quality(\textbf{this is where our work shines})
\subsection{Quality Measurement}

\subsubsection{Code Based Analysis}
\subsubsection{Efficiency}
\subsubsection{Maintainability}
\subsubsection{Other kinds of software Quality Measurement}

\section{Software Metric}
\subsection{Common Software Measurements}

\section{Program Analysis}
Program analysis is the process of automatically analyzing the behavior of computer programs. Two main approaches in program analysis are \textbf{static program analysis} and \textbf{dynamic program analysis}. Main applications of program analysis are program correctness and program optimization.
\subsection{Dynamic Program Analysis}
\subsection{Static Program Analysis}

\section{Quality Analysis Tools}
This section will list most used code quality analysis tools.

