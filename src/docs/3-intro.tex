% introducao
\chapter{Introduction}
 This chapter will drive the reader through the context and motivation of this work followed by the objectives and later the organization of this text is presented.  
 

\section{Context}

% stating how important quality is % 
One of the pillars of sustainable software development is its quality which can basically be defined as functional or non-functional where the first focuses on how the software meets its specification and how it works accordingly to its requirements and the second is aimed on how well the software is structured, we can generalize the first as being \emph{external quality} and second as \emph{internal quality}. To measure external quality there is the need to execute the software, also known as \emph{dynamic analysis}, either by an end user accessing the system or an automated process like for example functional testing or performance testing. There is no known way to assure functional quality without executing the software. Internal quality however can be verified by either \emph{statical analysis} that is mainly the inspection of the source code itself or by dynamic analysis which means executing the software like for example automated \emph{whitebox testing} which is the detailed
investigation of internal logic and structure of the code \citep{Khan 2012}.   

With good software quality in mind we take applications to another level where maintainability is increased, correctness is enhanced, defects are identified in early development stages which leads up to 100 times reduced costs \citep{Beohm 2001} and also other system characteristics like reusability, reliability and portability are benefited by higher software quality.  

% OSGi and its importance %
A well known and successful way to structure software architecture is to modularize its components. In the Java ecosystem although there is a moving to modularize the JDK and Java applications with the project Jigsaw \citep{Krill 2012} and also the recent \emph{microservices} movement \citep{Knorr 2014} for now the only practical working and well known solution for modular Java applications is OSGi, a component-based and service-oriented framework for building Java modular applications which is the \emph{de facto} standard solution for this kind of software since early 2000's. 

In the context of Java modular applications using OSGi and software quality there is no way to measure software internal quality which is the main objective of this work.          


\section{Objectives}

This work is focused on \emph{internal} OSGi projects quality mainly due to the following facts:

\begin{enumerate}
  \item There is no known standard way neither tools to measure OSGi projects internal quality \citep{Hamza 2013}.
  \item We already have tools and approaches to measure standard projects internal and external quality.
  \item For OSGi applications measuring \emph{external quality} the classical approaches like automated testing are sufficient and widely used.
\end{enumerate}
For measuring OSGi qualities first will be created the metrics based on good practices in the development of OSGi systems so in a second moment those metrics will be applied on top of real OSGi projects using a tool called \emph{Intrabundle} which was created during this work and also will be presented here. In the end the resulting output of Intrabundle and introspected projects qualities will be analyzed to conclude if created metrics have value for measuring Java modular applications or not.  

 
\section{Organization}

This text is organized in the following way. First chapter defines the context, motivation and objectives of this work. The second chapter will introduce the main concepts in the area of software quality like quality measurement, quality metrics, program analysis and quality analysis tools. The third chapter will present Java and OSGi, how standard Java and OSGi are different in respect to quality metrics and why we need different metrics for OSGi(TODO - depending it will be merged into chapter two). The fourth chapter presents Intrabundle, an OSGi code introspection tool to measure internal quality, we will see how Intrabundle works, what kind of information it extracts and what metrics it is applying. The fifth chapter will analyses the results Intrabundle produces and validate them to decide if this work has a valid contribution or not. The last chapter will present the conclusions and future work on this subject.