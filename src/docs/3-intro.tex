% introducao
\chapter{Introduction}
 This chapter will drive the reader through the context and motivation of this work followed by the objectives and later the organization of this text is presented.  
 

\section{Context}

% stating how important quality is % 
One of the pillars of sustainable software development is its quality which can basically be defined as internal and external where the first focuses on how software meets its specification and works accordingly to its requirements and the second is aimed on how well the software is structured and designed. To measure external quality there is the need to execute the software\footnote{Also known as dynamic analysis} either by an end user accessing the system or an automated process like for example functional testing or performance testing. Internal quality however can be verified by either \emph{statical analysis} that is mainly the inspection of the source code itself or by dynamic analysis which means executing the software like for example automated \emph{whitebox testing}, the detailed investigation of internal logic and structure of the code \citep{Khan 2012}.   

With good software quality in mind we take applications to another level where maintainability is increased, correctness is enhanced, defects are identified in early development stages, which can lead up to 100 times reduced costs \citep{Beohm 2001}, and also other characteristics like reusability, reliability and portability are benefited by higher software quality.  

% OSGi and its importance %
A well known and successful way to structure software architecture is to modularize its components allowing easier evolution of the system because smaller modules are typically easier to maintain than monolith applications. In the Java ecosystem there is a moving to modularize the JDK and Java applications with the project Jigsaw \citep{Krill 2012} and also a recent interest in \emph{microservices} \citep{Knorr 2014} arise. Although all this interest in modular application today the only practical working and well known solution for modular Java applications is OSGi, a very popular component-based and service-oriented framework for building Java modular applications. OSGi is the \emph{de facto} standard solution for this kind of software since early 2000's and have being used as basis of most JavaEE\footnote{A Java platform dedicated for enterprise applications which are usually secure and robust systems that display, manipulate and store large amounts of complex data maintained by an organization} application servers\footnote{Java application servers are like an extended virtual machine for running applications, transparently handling connections to the database, connections to the Web client, managing components like Enterprise Java Beans(EJB) and so on}, the open source IDE Eclipse, Atlassian Jira and Confluence to cite a few big players using OSGi. 

In the context of Java modular applications using OSGi and software quality there is no known standard way neither tools to measure OSGi projects \textit{internal quality} \citep{Hamza 2013} although for \emph{external quality} the classical approaches like automated testing are sufficient and widely used.

        
\section{Objectives}
The main objective of this work is to create software metrics to measure internal quality of OSGi based projects where this metrics must reflect good practices in the OSGi world. The main difference the proposed metrics have compared to classical software metrics is that the first will be based on modularity attributes that only exists in modular applications.   

Another aim of this work is to create a tool to apply and validate the metrics on real OSGi projects and finally analyze the resulting qualities produced by the tool.  

 
\section{Organization}

This text is organized in the following way. First chapter defines the context, motivation and objectives of this work. The second chapter will introduce the main concepts and technologies used in this work and will be divided into two main sections where the first will be focused in the area of software quality like quality measurement, quality metrics, program analysis and quality analysis tools and the second section of chapter two will present Java and OSGi, how standard Java and OSGi are different in respect to quality metrics and why we need different metrics for OSGi. The third chapter presents \textbf{Intrabundle}, an OSGi code introspection tool to measure internal quality, we will see how Intrabundle works, what kind of information it extracts and what metrics it is applying. The fourth chapter will analyze the results Intrabundle produces and validate them to decide if this work has a valid contribution or not. The last chapter will present the conclusions and future work on this subject.