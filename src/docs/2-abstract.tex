\begin{abstract}{}
 Todays software applications are becoming more complex, bigger, dynamic and harder to maintain. One way to overcome modern systems complexities is to build modular applications so we can divide it into small blocks which collaborate to solve bigger problems, the so called \emph{divide to conquer}. Another important aspect in the software industry that helps building large applications is the concept of software quality because its well known that higher quality softwares are easier to maintain and evolve at long term.

The Open Services Gateway Initiative(OSGi) is a very popular solution for building Java modular applications. It is very hard to measure the quality of OSGi systems due to its particular characteristics like service oriented, intrinsic modularity and component based approach.

In this work will be presented a tool called \emph{Intrabundle} that analyses OSGi projects and measure their internal quality. The tool extracts useful information that is specific to this kind of project and organize the analyzed data into Human readable reports in various formats. 

Yet it's also proposed 6 metrics based on good practices inside OSGi world which are applied to 10 real OSGi projects that vary in size, teams and domain.
\end{abstract}

