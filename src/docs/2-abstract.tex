\begin{abstract}{}
 Todays software applications are becoming more complex, bigger, dynamic and harder to maintain. One way to overcome nowadays complexities is to build modular applications so we can divide the problems into small blocks which collaborate to solve bigger problems, the so called \emph{divide to conquer}. Another important aspect in the software industry that helps building large applications is the notion of software quality because its well known that higher quality softwares are easier to maintain and evolve at long term.

The Open Services Gateway Initiative(OSGi) is the \emph{de facto} standard for building Java modular applications but there is no automated way to measure the quality of an OSGi system. In the context of Java applications there are many well known quality metrics to measure application's quality but when we move to Java modular applications where standard quality metrics does not fit or even exist we run out of options. 

In this work we will present a tool called \emph{intrabundle} which analyses OSGi applications and measure their quality. We also propose 6 metrics based on good practices inside OSGi world and apply them on top of 10 real OSGi projects which vary in size, teams and domain of solution.
\end{abstract}

