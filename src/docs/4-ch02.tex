\chapter{Basic Concepts}
This chapter presents an overview of the concepts and technologies that were studied and used on the development of this work. 
In section \textit{2.1 - Software Quality}, will be presented general aspects of software quality such as \textit{quality measurement},  \textit{software metrics}, \textit{program analysis} and some tools that are used in this area.  

Section \textit{2.2 - Java and OSGi} will introduce OSGi a framework for build service oriented Java modular applications. Finally section \ref{sec:forge} will introduce JBoss Forge, a Java framework used as runtime\footnote{Is software designed to support the execution of computer programs written in some computer language} for Intrabundle\footnote{A Java based project that will be presented later on this work}.   


\section{Software Quality}
\label{sec:quality}
There has been many definitions of software quality \citep[p. 23]{Kan 2002} and there is even an ISO norm for it, the ISO/IEC 25010 \citep{iso 2011}. All this definitions agree that the main motivation to perform continuous software quality management is to avoid \textbf{software failures} and increase \textbf{maintainability} in the sense that the more quality a program has the easier will be to maintain, the less bugs or abnormal behavior it will have and the more it will conform with its functional and non functional requirements\footnote{Functional and non functional requirements can be simply defined as \textit{what} the software does and \textit{how} the software will do respectively}. 

Software quality can be divided in two groups, the \textbf{external} and \textbf{internal} quality. 

\subsection{External Quality}
 When we talk about \textit{external quality} we are aiming to the user view which is the one that sees the software working and use it. This kind of quality is usually enforced through software testing. External quality can also be mapped to functional requirements so the greater external quality is the more usable and less defects it will have for example. 

 \subsection{Internal Quality}
 The opposite is internal or structural quality that aims to how the software is architect-ed internally which is the perspective of the programmer and non functional requirements. So the higher internal quality the better the code is structured, efficient, robust and maintainable it should be. Image 2.1 illustrates internal and external quality and its target audience.


\begin{figure}[h]
\caption{Internal and external quality audience}
\centering
\includegraphics[scale=0.5]{external-internal-quality}
\end{figure}
\FloatBarrier

% functional quality(performed via automated testing)
% structural quality(\textbf{this is where our work shines})

\subsection{Quality Measurement}
Quality measurement focuses on quantifying software desirable characteristics and each characteristic can have a set of measurable attributes, for example \textit{high cohesion} is a desirable characteristic and \textit{LOC - lines of code} is a measurable attribute related to cohesion. Quality measurement is close related to internal quality and in most cases is performed via static code analysis where program code is inspected to search for quality attributes to be measured but in some cases a dynamic analysis, where the program analysis is done during software execution, can be performed to measure characteristics that can be perceived only when software is running, for example performance or code coverage\footnote{A technique that measures the code lines that are executed for a given set of software tests, its also considered a software metric.}.     

In the extent of this work the characteristics of software to be considered and measured later are listed and described in table 2.1:  

\begin{table}[h]
\caption{Quality characteristics to be considered}
\begin{center}
    \begin{tabular}{  p{3cm} | p{8cm} | p{5cm} }
    \Xhline{2\arrayrulewidth}
    Characteristic & Description & OSGi example \\  \hline
    Reliability & the degree to which a system or component performs its required functions under stated conditions for a specified period of time. & Bundles should not have stale service references.\\ \hline
    Performance Efficiency & Performance relative to the amount of resources used under stated conditions for a specified
period of time. & Bundle startup time, also bundle dependency can decrease performance. \\ \hline
    Security & the degree of protection of information and data so that unauthorized persons or systems cannot read, access or modify them. & Bundles should declare permissions \\ \hline
    Maintainability & The degree to which the product can be modified. & Modules should be loosely coupled, bundles should publish only interfaces etc. \\ 
    \Xhline{2\arrayrulewidth}
     
    \end{tabular}
    Source: \cite{cisq 2012}
\end{center}
\end{table}
\FloatBarrier

\subsection{Software Metrics}
\label{sec:soft_metrics}
A software metric is the measurement of a software attribute which in turn is a quantitative calculation of a characteristic. Software metrics can be classified into three categories: product metrics\footnote{Product metrics describe the characteristics of the product such as size, complexity, design features, performance}, process metrics\footnote{Process metrics can be used to improve software development and maintenance.Examples include the effectiveness of defect removal during development and response time of bug fixing}, and project metrics\footnote{Project metrics describe the project characteristics and execution. Examples include the number of
software developers, cost, schedule, and productivity}. Software quality metrics are a subset of software metrics that focus on the quality aspects of the product, process, and project \citep{Kan 2002}.


\subsubsection{Good Software Metrics}
Good metrics may have the following aspects:

\begin{itemize}
  \item \emph{Linear}: metric values should follow an intuitive way to compare its values like for example higher values should correspond to better quality whereas lower values to worse quality and vice versa. 
  \item \emph{Independent}: two metric values should not interfere on each other.
  \item \emph{Repeatable}: this is a very important aspect in continuous quality management where software is changing all the time and we want to measure quality on every change.
  \item \emph{Accurate}: the metric should be meaningful and should help answer how good a software attribute is, for example using latency\footnote{The delay incurred in communicating a message, the time the message spends “on the wire”} to calculate response time\footnote{The total time it takes from when a user makes a request until they receive a response} in a web application isn't accurate.       
\end{itemize}


\subsubsection{Common Software Metrics}
The table \ref{common metrics} below shows some well known software metrics and its description:

\begin{table}[h]
\caption{Common Software metrics}
\label{common metrics}
\begin{center}
    \begin{tabular}{  p{3cm} | p{8cm} }
    \hline
    Metric & Description \\  \hline
    Cyclomatic complexity & It is a quantitative measure of the complexity of programming instructions.\\ \hline
    Cohesion & measure the dependency between units of code like for example classes in object oriented programing or modules in modular programming like OSGi. \\ \hline
    Coupling & measures how well two software components are data related or how dependent they are. \\ \hline
    Lines of code (LOC) & used to measure the size of a computer program by counting the number of lines in the text of the program's source code. \\ \hline
    Code coverage & measures the code lines that are executed for a given set of software tests \\ \hline
    Function point analysis (FPA) & used to measure the size (functions) of software. \\
    \hline

    \end{tabular}
    \\*Source: \cite{sqa 2012}
\end{center}
\end{table}

\FloatBarrier


\subsection{Program Analysis}
Program analysis is the process of automatically analyzing the behavior of computer programs. Two main approaches in program analysis are \textbf{static program analysis} and \textbf{dynamic program analysis}. Main applications of program analysis are program correctness, program optimization and quality measurement.

\subsubsection{Static Program Analysis}
 Is the analysis of computer software that is performed without actually executing programs \citep{Wichmann 1995}. In this kind of analysis source code is inspected and valuable information is collected based on its internal structure and components. 

\subsubsection{Dynamic Program Analysis}
Is a technique that analyze the system's behavior on the fly, while it is executing. The main objectives of this kind of analyze is to catch \textit{memory leaks}\footnote{Resources that are hold on system's memory and aren't released}, identify arithmetic errors and extract code coverage and measure performance. 


\subsection{Quality Analysis Tools}
\label{ch2:qatools}
The table \ref{quality tools} lists some code quality analysis tools in the Java ecosystem:

\begin{table}[h]
\caption{Quality analysis tools}
\label{quality tools}
\begin{center}
    \begin{tabular}{  p{3cm} | p{8cm} | p{2cm} }
    \Xhline{2\arrayrulewidth}
    Name & Description & Type\\  \hline
    \href{http://www.sonarqube.org}{SonarQube} &  An open source platform for continuous inspection of code quality. & static\\ \hline
    \href{http://findbugs.sourceforge.net/}{FindBugs} & An open-source static bytecode analyzer for Java. & static\\ \hline 
    \href{http://checkstyle.sourceforge.net/}{Checkstyle} & A static code analysis tool used in software development for checking if Java source code complies with coding rules. & static\\ \hline 
    \href{http://pmd.sourceforge.net/}{PMD} & A static ruleset based Java source code analyzer that identifies potential problems. & static\\ \hline 
    \href{http://www.contemplateltd.com/threadsafe}{ThreadSafe} & A static analysis tool for Java focused on finding concurrency bugs. & static\\ \hline
    \href{http://www.intooitus.com/products/infusion}{InFusion} & Full control of architecture and design quality. & static\\ \hline 
    \href{https://www.ej-technologies.com/products/jprofiler/overview.html}{JProfiler} & helps you resolve performance bottlenecks, pin down memory leaks and understand threading issues & dynamic \\ \hline
    \href{http://www.eclemma.org/jacoco/}{JaCoCo} & A free code coverage library for Java. & dynamic\\ \hline 
    \href{https://code.google.com/p/javamelody/}{Javamelody} & Java or Java EE application Monitoring in QA and production environments. & dynamic\\ \hline 
    \href{http://www-304.ibm.com/partnerworld/gsd/solutiondetails.do?solution=23517&expand=true}{Introscope} & An application management solution that helps enterprises keep their mission-critical applications high-performing and available 24x7. & dynamic\\ 
    \Xhline{2\arrayrulewidth}

    \end{tabular}
    %\\*Source: \cite{sqa 2012}
\end{center}
\end{table}

\FloatBarrier

Figure \ref{pmd_violation} shows the execution of static analysis on \textit{Intrabundle} using \emph{PMD}, note that it is based on rules and Intrabundle break some of them(intentionally) like \emph{Unused variables}, \emph{EmptyCatchBlock} so PMD consider them compile failure and the project cannot be compiled until the rules are fixed in code:

\begin{figure}[h]
\caption{Intrabundle PMD rule violation}
\label{pmd_violation}
\centering
\includegraphics[scale=0.5]{intrabundle-pmd-build-failure}
\end{figure}

\FloatBarrier

The rules are totally customizable via xml configuration, Intrabundle PMD rules are shown in Figure \ref{pmd_ruleset}:

\begin{figure}[h]
\caption{Intrabunde PMD ruleset}
\label{pmd_ruleset}
\centering
\includegraphics[scale=0.5]{intrabundle-pmd-ruleset}
\\*Source: \cite{intrabundle pmd 2014}
\end{figure}

\FloatBarrier

\section{Java and OSGi}

In the context of Java™ programming language \citep{Arnold 2005}, which accordingly to IEEE spectrum of this year is the most popular programming language \citep{ieee spectrum 2014}, and modular applications\footnote{A software design technique that emphasizes separating the functionality of a program into independent, interchangeable modules which represent a separation of concerns and improves maintainability} this section will introduce the Java language and OSGi framework.

\subsection{The Java language}
Java is a general purpose object oriented\footnote{Object-oriented programming(OOP) integrates code and data using the concept of an "object" which is a piece of software that holds state and behavior} programming language created by Sun Microsystems in 1995 which aims on simplicity, readability and universality. Java runs on top of the so called JVM, the acronym for Java Virtual Machine, which is an abstract computing machine\footnote{Also known as \textit{Virtual Machine} which is an emulation of a particular computer system} and platform-independent execution environment that execute Java byte code\footnote{The intermediate output of the compilation of a program written in Java that can be read by the JVM}. The JVM converts java byte code into host machine language(e.g. linux, windows etc...) allowing Java programs to "run everywhere" independently of operating system or platform. JVM implementations are different for each platform but the generated bytecode is the same, Figure \ref{jvm} illustrates how JVM works:

\begin{figure}[h]
\caption{JVM architecture}
\label{jvm}
\centering
\includegraphics[scale=0.5]{jvm}
\end{figure}  
\FloatBarrier

Other aspects of Java are listed below:   

\begin{itemize}
\item Type safe\footnote{Type safety is the extent to which a programming language discourages or prevents type errors} 
\item Dynamic: during the execution of a program, Java can dynamically load classes 
\item Strong memory management(no explicit pointer)
\item Automatic garbage collection to release unused objects from memory
\item Robust: extensive compile-time checking so bugs can be found early
\item Multithreaded\footnote{Multithreading is a program’s capability to perform several tasks simultaneously}
\item Distributed: networking capability is inherently integrated into Java
\end{itemize}

\subsection{The OSGi service platform}
OSGi is a component based service oriented platform specification maintained by \emph{OSGi Alliance}\footnote{A non profit worldwide consortium of technology innovators} that runs on top of Java. As of November 2014 the specification is at version 6 and currently has four implementations\footnote{\href{http://felix.apache.org}{Apache Felix}, \href{http://eclipse.org/equinox/}{Eclipse Equinox}, \href{http://www.knopflerfish.org}{Knopflerfish} and \href{http://www.prosyst.com}{ProSyst}}. It is composed by \emph{OSGi framework} and \emph{OSGi standard services}. The framework is the runtime that provides the basis of all OSGi module system functionalities like modules management for example. Standard services define some reusable apis and extension points to easy development of OSGi based applications. Figure \ref{osgi architecture} illustrates OSGi platform architecture:

\begin{figure}[h]
\caption{OSGi architecture}
\label{osgi architecture}
\centering
\includegraphics{osgi-architecture}
\end{figure}  
\FloatBarrier



\subsubsection{Bundles}
Bundles are the building blocks of OSGi applications. A bundle\footnote{Also known as module} is a group of Java classes and resources packed as .jar extension with additional metadata in manifest MANIFEST.MF file describing its module boundaries like for example the packages it imports and exports. Below is an OSGi manifest file example:

\begin{lstlisting}
Bundle-Name: Hello World
Bundle-SymbolicName: org.wikipedia.helloworld
Bundle-Description: A Hello World bundle
Bundle-ManifestVersion: 2
Bundle-Version: 1.0.0
Bundle-Activator: org.wikipedia.Activator
Export-Package: org.wikipedia.helloworld;version="1.0.0"
Import-Package: org.acme.api;version="1.1.0"
\end{lstlisting} 
\FloatBarrier

Looking at manifest OSGi can ensure its most important aspect, \emph{modularity}, so for example our \textbf{Hello World} bundle will only be started(later we will explore bundle lifecycle) if and only if there is a bundle (in resolved or installed state) that exports \emph{org.acme.api} package, this is called \textbf{explicit boundaries}.

With OSGi, you modularize applications into bundles. Each bundle is a tightly coupled, dynamically loadable collection of classes packed in JARs\footnote{acronym for Java Archive, a file that used to aggregate many Java class files and associated metadata and resources (text, images, etc.) into one file to distribute}, and configuration files that explicitly declare any external dependencies. All these characteristics are provided in OSGi by three conceptual layers that will be briefly presented here, \emph{Module}, \emph{Lifecycle} and \emph{Service}.

\subsubsection{Module layer}
This layer is the basis for others as modularization is the key concept of OSGi. The module layer defines OSGi module concept - bundle, which is a JAR file with extra metadata. It also handles the packaging and sharing of Java packages between bundles and the hiding of packages from other bundles. The OSGi framework dynamically resolves dependencies among bundles and performs bundle resolution to match imported and exported packages. This layer ensures that class loading happens in a consistent and predictable way.

\begin{figure}[h]
\caption{Module Layer}
\label{module layer}
\centering
\includegraphics{module-layer}
\\*Source: \cite{conceptual layers 2011}
\end{figure}
\FloatBarrier

\subsubsection{Lifecycle layer}
Provides access to the underlying OSGi framework through the \emph{Bundle Context} object. This layer handles the lifecycle of individual bundles so you can manage your application dynamically, including starting and stopping bundles to manage and evolve them over time. Bundles can be dynamically installed, started, updated, stopped and uninstalled. Figure \ref{bundle lifecycle} shows bundle lifecycle and its possible states where transitions are performed by OSGi commands like \emph{start} or \emph{stop} for example and states are represented in squares:

\begin{figure}[h]
\caption{OSGi bundle Lifecycle}
\label{bundle lifecycle}
\centering
\includegraphics{bundle-lifecycle}
\end{figure}
\FloatBarrier

If OSGi were a car, module layer would provide modules such as tire, seat, etc, and the lifecycle layer would provide electrical wiring which makes the car run. 

\begin{figure}[h]
\label{lifecycle layer}
\caption{Lifecycle Layer}
\centering
\includegraphics{lifecycle-layer}
\\*Source: \cite{conceptual layers 2011}
\end{figure}
\FloatBarrier

\subsubsection{Service layer}
This layer provides communication among modules and their contained components. Service providers publish services\footnote{A Service is an operation offered as an interface that stands alone in the model, without encapsulating state \citep{Evans 2003}} to \emph{service registry}, while service clients search the registry to find available services to use. The registry is accessible to all bundles so they can \emph{publish} its services as well \emph{consume} services from other bundles.  

This is like a service-oriented architecture (SOA) which has been largely used in web services. Here OSGi services are local to a single VM, so it is sometimes called SOA in a VM.

\begin{figure}[h]
\label{service layer}
\caption{Service Layer}
\centering
\includegraphics[scale=0.5]{service-layer}
\end{figure}
\FloatBarrier


\subsection{Vanilla Java vs OSGi}
The main motivation behind OSGi and advantage over standard Java application, as illustrated before, is the modularity. The main issue with Java default runtime is the way Java classes are loaded, it is the root cause that inhibits modularity in classical Java applications. In standard Java, user classes\footnote{Classes that are defined by developers and third parties and that do not take advantage of the extension mechanism} are loaded by a classloader\footnote{A class loader is an object that is responsible for loading classes} from the same classpath\footnote{classpath tells Java virtual machine where to look in the filesystem for files defining these classes} which is commonly referred as a \emph{flat classpath}. A flat classpath is the main cause of a well known problem in Java applications, the \emph{Jar Hell}\footnote{A term used to describe all the various ways in which the classloading process can end up not working}. Figure \ref{jar hell} is an example of Jar hell where multiple JARs containing overlapping classes(consider each shape as being a Java class) and/or packages are merged based on their order of appearance in the class path.


\begin{figure}[h]
\caption{Java jar hell}
\label{jar hell}
\centering
\includegraphics[scale=0.5]{jar-hell}
\\*Source: \cite[p. 7]{Hall 2011}
\end{figure}  
\FloatBarrier


 In the OSGi environment instead of a \emph{flat classpath} each bundle has its classloader and its classpath. See Figure \ref{bundle classpath} where Bundle A’s classpath is defined as the union of its bundle classpath with its imported packages, which are provided by bundle B’s exports.


\begin{figure}[h]
\caption{Bundle classpath}
\label{bundle classpath}
\centering
\includegraphics[scale=0.5]{bundle-classpath}
\\*Source: \cite[p. 59]{Hall 2011}
\end{figure}  
\FloatBarrier

In OSGi runtime we can say we have a graph of classpaths that allows powerful versioning mechanisms so for example we can have multiple versions of the same class or resource loaded at the same time(used by different bundles). This enables independent evolution of dependent artifacts which, in the Java world, is unique to OSGi environments \citep{semantic versioning 2010}.   

\section{JBoss Forge}
\label{sec:forge}
\subsection{Introduction}
\href{http://forge.jboss.org/1.x/}{JBoss Forge} is a modular\footnote{Forge runtime is based on \href{https://docs.jboss.org/author/display/MODULES/Introduction}{JBoss modules} which is a technology, like OSGi but not so popular, for building modular applications} plugin based general purpose command line tool(CLI). Forge can be started through command line or be integrated in an IDE. Figure \ref{forge-cli} shows Forge initial screen:

\begin{figure}[h]
\caption{Forge initial screen}
\label{forge-cli}
\centering
\includegraphics[scale=0.5]{forge-cli}
\end{figure}  
\FloatBarrier

Forge runs on any operating system that can run Java and have built in startup scripts\footnote{Executable files that initiate a process or system} for Windows, Linux and MacOS.

\subsection{Forge Plugin}
\label{sec:forge:plugin}
A Forge plugin can be as simple as a tool to print files to the console, or as complex as deploying an application to a server, tweeting\footnote{Term used to refer as publishing messages to the twitter social media.} the status of your latest source-code commit, or even sending commands to a home-automation system. The sky is the limit!. 

Every plugin offers a set o commands that may be restricted by a facet.

\subsubsection{Example}
Below is a simple Forge plugin named \emph{hello-world} with a command named \emph{sayHello} that prints "Hello World" when executed:

\begin{lstlisting}[language=java,label=forge-plugin,caption=Forge plugin example]
@Alias("hello-world")
public class HelloWorldPlugin implements Plugin {

    @Command(value = "sayHello")
    public void countBundles(PipeOut out) {
        out.println("Hello World" );
    }

}
\end{lstlisting}

Plugin is just a marker interface so Forge can identify plugins. To fire the sayHello command one have to start forge, install the HelloPlugin and then can use the command by typing \emph{hello-world sayHello} in Forge console and so "Hello World" should be printed in console.

\subsection{Facets}
\label{sec:forge:facet}
A Facet in the Forge environment is responsible for restricting the usage of a plugin. It is in fact an interface\footnote{In object oriented programming is a contract that defines which methods the implementors of the interface must provide} with a method with return type boolean that must decide if the facet is installed.

\subsubsection{Example}
Below is an example of facet that restricts the usage of hello-world plugin, in the example the command should be only available when user is in a directory named \emph{hello} otherwise Forge will claim that the command does not exist in current context.

\begin{lstlisting}[language=java,label=hello-facet,caption=Forge facet example]

public class HelloFacet implements Facet {

   @Inject 
   Project project;

   @Override
   public boolean isInstalled() {
      return project.getProjectRoot().getName().equals("hello");
    }
}
\end{lstlisting}

So the idea of a facet is that it is active when isInstalled method return true. In case of HelloFacet only when user current directory is
named "hello". To get user current directory we ask forge, through dependency injection, for the current project. Project is a Java object that holds information of the current user project like its directory. 

To activate the facet we must annotate HelloWorld plugin with RequiresFacet:

\begin{lstlisting}[language=java,label=forge-plugin-facet,caption=Forge plugin with facet example]
@Alias("hello-world")
@RequiresFacet(HelloFacet.class)
public class HelloWorldPlugin implements Plugin {

    @Command(value = "sayHello")
    public void countBundles(PipeOut out) {
        out.println("Hello World" );
    }
}
\end{lstlisting}

\subsection{Project Locator}
\label{sec:forge:locator}
 A project locator is a component responsible for creating Java objects that represent useful information in the forge runtime, they are called \emph{project} in forge. Forge calls all locators available when user change directory in command line and the first locator that is matched will create a Java object representing the current Project. Its the same idea of facets but instead of restricting plugin commands it creates object and made them available for Forge runtime. That was how we could inject current user project in HelloFacet before.

\subsection{Applications}
Forge can be used as a command line tool or integrated in main IDEs like Eclipse, Netbeans or IntelliJ. To be used as command line tool one must download a zip distribution containing a forge executable that runs on main operating systems\footnote{As forge runs on top of Java, Forge inherits its \emph{universality}}.    

Forge has an important role on this work as it was the ground for creating Intrabundle, a tool based on forge runtime that will be introduced later.